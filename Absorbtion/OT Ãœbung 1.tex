\documentclass[11pt]{scrartcl}
\usepackage[T1]{fontenc}
\usepackage[a4paper, left=3cm, right=2cm, top=2cm, bottom=2cm]{geometry}
\usepackage[activate]{pdfcprot}
\usepackage[ngerman]{babel}
\usepackage[parfill]{parskip}
\usepackage[utf8]{inputenc}
\usepackage{kurier}
\usepackage{amsmath}
\usepackage{amssymb}
\usepackage{xcolor}
\usepackage{epstopdf}
\usepackage{txfonts}
\usepackage{fancyhdr}
\usepackage{graphicx}
\usepackage{prettyref}
\usepackage{hyperref}
\usepackage{eurosym}
\usepackage{setspace}
\usepackage{units}
\usepackage{eso-pic,graphicx}
\usepackage{icomma}
\usepackage{pdfpages}

\definecolor{darkblue}{rgb}{0,0,.5}
\hypersetup{pdftex=true, colorlinks=true, breaklinks=false, linkcolor=black, menucolor=black, pagecolor=black, urlcolor=darkblue}



\setlength{\columnsep}{2cm}


\newcommand{\arcsinh}{\mathrm{arcsinh}}
\newcommand{\asinh}{\mathrm{arcsinh}}
\newcommand{\ergebnis}{\textcolor{red}{\mathrm{Ergebnis}}}
\newcommand{\fehlt}{\textcolor{red}{Hier fehlen noch Inhalte.}}
\newcommand{\betanotice}{\textcolor{red}{Diese Aufgaben sind noch nicht in der Übung kontrolliert worden. Es sind lediglich meine Überlegungen und Lösungsansätze zu den Aufgaben. Es können Fehler enthalten sein!!! Das Dokument wird fortwährend aktualisiert und erst wenn das \textcolor{black}{beta} aus dem Dateinamen verschwindet ist es endgültig.}}
\newcommand{\half}{\frac{1}{2}}
\renewcommand{\d}{\, \mathrm d}
\newcommand{\punkte}{\textcolor{white}{xxxxx}}
\newcommand{\p}{\, \partial}
\newcommand{\dd}[1]{\item[#1] \hfill \\}

\renewcommand{\familydefault}{\sfdefault}
\renewcommand\thesection{}
\renewcommand\thesubsection{}
\renewcommand\thesubsubsection{}


\newcommand{\themodul}{Optische Technologie - Fragen zur Absorbtion}
\newcommand{\thetutor}{Prof. Rateike}
\newcommand{\theuebung}{Praktikum}

\pagestyle{fancy}
\fancyhead[L]{\footnotesize{C. Hansen}}
\chead{\thepage}
\rhead{}
\lfoot{}
\cfoot{}
\rfoot{}

\title{\themodul{}, \theuebung{}, \thetutor}


\author{Christoph Hansen \\ {\small \href{mailto:chris@university-material.de}{chris@university-material.de}} }

\date{}


\begin{document}

\maketitle

Dieser Text ist unter dieser \href{http://creativecommons.org/licenses/by-nc-sa/4.0/}{Creative Commons} Lizenz veröffentlicht.

\textcolor{red}{Ich erhebe keinen Anspruch auf Vollständigkeit oder Richtigkeit. Falls ihr Fehler findet oder etwas fehlt, dann meldet euch bitte über den Emailkontakt.}

\tableofcontents


\newpage

\section{Frage 1}


\section{Frage 2}

Extinktion wird auch optische Dichte genannt und ist die wahrnehmungsgerecht logarithmisch formulierte Opazität $O$. Sie ist also ein Maß für die Abschwächung einer Strahlung nach durchqueren eines Mediums.


\section{Frage 3}

Das Lambert-Beersche Gesetz gleicht einem Zerfallsgesetz und lautet:

\begin{align*}
I(r) &= I_0 \cdot e^{-\mu r} 
\end{align*}

$\mu$ beschreibt die optischen Eingenschaften des durchquerten Material und $r$ wo im Medium betrachtet wird.


\section{Frage 4}

Im Prinzip braucht man nur eine möglichst punktförmige Lichtquelle, ein Gefäß in dass das zu untersuchende Medium eingefüllt wird und ein Gerät zu Messen der Intensität nach durchqueren den Mediums.


\section{Frage 5}

In das Lasertechnikskript gucken (steht auch auf der Website ;) )


\section{Frage 6}

\subsection*{Aufbau}

Eine typische Silizium-Photodiode besteht aus einem schwach n-dotierten Grundmaterial mit einer stärker dotierten Schicht auf der Rückseite, die den einen Kontakt (Kathode) bildet. Die Lichtempfindliche Fläche wird definiert durch einen Bereich mit einer dünnen p-dotierten Schicht an der Vorderseite. Diese Schicht ist dünn genug damit das meiste Licht bis zum p-n-Übergang gelangen kann. Der elektrische Kontakt ist meistens am Rand.[3] Auf der Oberfläche ist eine Schutzschicht als Passivierung und Antireflexionsschicht. Oft befindet sich vor der Photodiode zusätzlich ein lichtdurchlässiges Schutzfenster oder sie befindet sich in transparentem Vergussmaterial.

\subsection*{Funktion}

Treffen Photonen ausreichender Energie auf das Material der Diode, so werden Ladungsträger (Elektron-Loch-Paare) erzeugt. In der Raumladungszone driften die Ladungsträger schnell entgegen der Diffusionsspannung in die gleichartig dotierten Zonen, und führen zu einem Strom. Außerhalb der Raumladungszone erzeugte Ladungsträger können auch zum Strom beitragen. Sie müssen aber erst per Diffusion bis zur Raumladungszone gelangen. Dabei geht ein Teil durch Rekombination verloren und es entsteht eine kleine Verzögerung.[4] Ohne externe Verbindung der Anschlüsse entsteht an diesen eine messbare Spannung gleicher Polarität wie die Durchflussspannung (Sättigung). Sind die Anschlüsse miteinander elektrisch verbunden oder befinden sie sich an einer Spannung in Sperrrichtung der Diode, fließt ein Photostrom, der proportional zum Lichteinfall ist.


\section{Frage 7}

Bei Gleichlichtmessungen stört der Nullpunktdrift, der unter anderem durch die Änderung der strahlenden Umgebung verursacht wird.

\section{Frage 8}


Die Wechsellichtmethode nutzt die bei Photoleitern bestehende Möglichkeit aus, die Elektronenkonzentration unabhängig von der Beweglichkeit durch variable Lichteinstrahlung zu verändern. Man versieht, indem man mit periodisch moduliertem Licht arbeitet, die Elektonenkonzentration mit deiner Art Merkmal, mittels dessen sie von der Leitfähigkeit $\sigma = e \cdot n_e \cdot b_e$ außerdem noch auftretenden Beweglichkeit ($b_e$) abtrennbar wird. 

Diese Methode wird genutzt, weil man 

\begin{itemize}
	\item eine Unterscheidung gegenüber der Umgebungsstrahlung zu haben (periodische 
	Nullpunktsetzung, wichtig im infraroten Spektralbereich)
	\item das Signal leichter elektronisch verarbeiten zu können. Die Verstärkung der Trägerfrequenz 
	erfolgt schmalbandig mit anschließender phasenempfindlicher Gleichrichtung und 
	Glättung (sog. Lock in - Verfahren)
\end{itemize}


\section{Frage 9 + 10}

siehe 8







\end{document}