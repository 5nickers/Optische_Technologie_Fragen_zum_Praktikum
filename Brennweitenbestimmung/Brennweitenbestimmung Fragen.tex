\documentclass[11pt]{scrartcl}
\usepackage[T1]{fontenc}
\usepackage[a4paper, left=3cm, right=2cm, top=2cm, bottom=2cm]{geometry}
\usepackage[activate]{pdfcprot}
\usepackage[ngerman]{babel}
\usepackage[parfill]{parskip}
\usepackage[utf8]{inputenc}
\usepackage{kurier}
\usepackage{amsmath}
\usepackage{amssymb}
\usepackage{xcolor}
\usepackage{epstopdf}
\usepackage{txfonts}
\usepackage{fancyhdr}
\usepackage{graphicx}
\usepackage{prettyref}
\usepackage{hyperref}
\usepackage{eurosym}
\usepackage{setspace}
\usepackage{units}
\usepackage{eso-pic,graphicx}
\usepackage{icomma}
\usepackage{pdfpages}

\definecolor{darkblue}{rgb}{0,0,.5}
\hypersetup{pdftex=true, colorlinks=true, breaklinks=false, linkcolor=black, menucolor=black, pagecolor=black, urlcolor=darkblue}



\setlength{\columnsep}{2cm}


\newcommand{\arcsinh}{\mathrm{arcsinh}}
\newcommand{\asinh}{\mathrm{arcsinh}}
\newcommand{\ergebnis}{\textcolor{red}{\mathrm{Ergebnis}}}
\newcommand{\fehlt}{\textcolor{red}{Hier fehlen noch Inhalte.}}
\newcommand{\betanotice}{\textcolor{red}{Diese Aufgaben sind noch nicht in der Übung kontrolliert worden. Es sind lediglich meine Überlegungen und Lösungsansätze zu den Aufgaben. Es können Fehler enthalten sein!!! Das Dokument wird fortwährend aktualisiert und erst wenn das \textcolor{black}{beta} aus dem Dateinamen verschwindet ist es endgültig.}}
\newcommand{\half}{\frac{1}{2}}
\renewcommand{\d}{\, \mathrm d}
\newcommand{\punkte}{\textcolor{white}{xxxxx}}
\newcommand{\p}{\, \partial}
\newcommand{\dd}[1]{\item[#1] \hfill \\}

\renewcommand{\familydefault}{\sfdefault}
\renewcommand\thesection{}
\renewcommand\thesubsection{}
\renewcommand\thesubsubsection{}


\newcommand{\themodul}{Optische Technologie - Fragen zur Absorbtion}
\newcommand{\thetutor}{Prof. Rateike}
\newcommand{\theuebung}{Praktikum}

\pagestyle{fancy}
\fancyhead[L]{\footnotesize{C. Hansen}}
\chead{\thepage}
\rhead{}
\lfoot{}
\cfoot{}
\rfoot{}

\title{\themodul{}, \theuebung{}, \thetutor}


\author{Christoph Hansen \\ {\small \href{mailto:chris@university-material.de}{chris@university-material.de}} }

\date{}


\begin{document}

\maketitle

Dieser Text ist unter dieser \href{http://creativecommons.org/licenses/by-nc-sa/4.0/}{Creative Commons} Lizenz veröffentlicht.

\textcolor{red}{Ich erhebe keinen Anspruch auf Vollständigkeit oder Richtigkeit. Falls ihr Fehler findet oder etwas fehlt, dann meldet euch bitte über den Emailkontakt.}

\tableofcontents


\newpage

\section{Frage 1}

\begin{align*}
\frac{1}{f} = \frac{1}{g} + \frac{1}{b}
\end{align*}


\section{Frage 2}

Die Vergrößerung setzt die Größe des Bildes in Zusammenhang mit der Größe des Gegenstandes. Man kann sie über folgende Formel berechnen:

\begin{align*}
m = \frac{B}{G} = - \frac{b}{g}
\end{align*}


\section{Frage 3}

So wie man es schon in der Schule gelernt hat, über die Hauptstrahlen......


\section{Frage 4}

\begin{array}{ll}
\text{bikonvex:}	& \text{die Linse ist beidseitig nach außen gewölbt} \\ 
\text{plankonvex:}	& \text{die Linse ist nur auf einer Seite nach außen gewölbt} \\ 
\text{bikonkav:}	& \text{die Linse ist beidseitig nach innen gewölbt} \\ 
\text{plankonkav:}	& \text{die Linse ist nur auf einer Seite nach innen gewölbt} 
\end{array} 


\section{Frage 5}

siehe Seite 3 der Anleitung zu diesem Versuch.


\section{Frage 6}

Die Strahlen mit denen konstruiert wird, also Hauptstrahl....


\section{Frage 7 + 8}

Eine Sammellinse wir auch positive Linse genannt, während eine Zerstreuungslinse auch negative Linse genannt wird.


\section{Frage 9}

Das geht mit der Matrizenoptik sehr schön:

\begin{align*}
\left[
\begin{array}{cc}
1 & 0 \\ 
\frac{1}{-f_2}& 1 
\end{array} 
\right]
\cdot
\left[
\begin{array}{cc}
1 & d \\ 
0 & 1 
\end{array} 
\right]
\cdot
\left[
\begin{array}{cc}
1 & 0 \\ 
\frac{1}{-f_1}& 1 
\end{array} 
\right]
\end{align*}


\end{document}